\documentclass[10pt,a4paper]{report}
\usepackage[utf8]{inputenc}
\usepackage{amsmath}
\usepackage{amsfonts}
\usepackage{amssymb}
\begin{document}

\begin{flushleft}
\underline{Notation}\\
\vspace{0.1cm}
\textbf{Y} $\equiv$ One-dimensional vector containing the outcome measure of interest for each unit of observation; i.e., a cross-sectional measurement of forest cover change. $Y_{i}$ represents the outcome measurement at unit of observation \textit{i}. 
\vspace{0.25cm}

\textbf{X} $\equiv$ j by i matrix containing ancillary information which may impact the outcome measure of interest, excluding the treatment.  $X_{j,i}$ represents the information for covariate \textit{j} at unit of observation \textit{i}.
\vspace{0.25cm}

\textbf{T} $\equiv$ One-dimensional vector containing the treatment status for each unit of observation; i.e., if a project to decrease deforestation existed at that location.  $T_{i}$ represents the treatment status at unit of observation \textit{i}. \vspace{0.25cm}

\end{flushleft}
A data generation process is run multiple times to examine the impact of spatial spillover in any of the three elements (\textbf{Y,X,T}) defined above in causal inferential designs seeking to isolate the impact of a treatment on a given outcome of interest.


\begin{equation}
$$X$$
\end{equation}



\end{document}