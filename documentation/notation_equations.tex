\documentclass[10pt,a4paper]{report}
\usepackage[utf8]{inputenc}
\usepackage{amsmath}
\usepackage{amsfonts}
\usepackage{amssymb}
\begin{document}

\begin{flushleft}
\underline{Notation}\\
\vspace{0.1cm}
\textbf{Y} $\equiv$ One-dimensional vector containing the outcome measure of interest for each unit of observation; i.e., a cross-sectional measurement of forest cover change. $Y_{i}$ represents the outcome measurement at unit of observation \textit{i}. 
\vspace{0.25cm}

\textbf{X} $\equiv$ j by i matrix containing ancillary information which may impact the outcome measure of interest, excluding the treatment.  $X_{j,i}$ represents the information for covariate \textit{j} at unit of observation \textit{i}.
\vspace{0.25cm}

\textbf{T} $\equiv$ One-dimensional vector containing the treatment status for each unit of observation; i.e., if a project to decrease deforestation existed at that location.  $T_{i}$ represents the treatment status at unit of observation \textit{i}. \vspace{0.25cm}

\end{flushleft}
Here, we  examine the impact of spatial spillover in any of the three elements \textbf{Y,X,T} in matching methods for causal inference.  Spatial spillover can be simply understood as the presence of correlation between measurements in neighboring units.  For example:
\begin{enumerate}
\item \textbf{Y} - in the case of deforestation,  communities that practice lumber harvesting as a livelihoods strategy may suggest that to neighboring communities, or even enter into neighboring lands to conduct lumber operations.
\item \textbf{X} - population could very reasonably be inferred to be a driver of deforestation; neighbouring communities may have similar populations due to cultural demographic characteristics.
\item \textbf{T} - a treatment - i.e., paying a community to prevent deforestation could result in an increased likelihood of funding neighboring communities due to word-of-mouth, or treatment designs which drive down costs by reducing travel times.
\end{enumerate}
These are just illustrative examples, but highlight the potential challenge that modeling spillover effects can cause.  

\begin{equation}
$$X$$
\end{equation}



\end{document}